\hypertarget{index_preface_sec}{}\subsection{\-Preface\-: How to use this document}\label{index_preface_sec}
\-The documentation for \-Force\-Balance exists in two forms\-: a web page and a \-P\-D\-F manual. \-They contain equivalent content. \-The newest versions of the software and documentation, along with relevant literature, can be found on the \href{https://simtk.org/home/forcebalance/}{\tt \-Sim\-T\-K website}.

{\bfseries \-Users} of the program should read the {\itshape \-Introduction, \-Installation\/}, {\itshape \-Usage\/}, and {\itshape \-Tutorial\/} sections on the main page.

{\bfseries \-Developers and contributors} should read the \-Introduction chapter, including the {\itshape \-Program \-Layout\/} and {\itshape \-Creating \-Documentation\/} sections. \-The {\itshape \href{http://leeping.github.io/forcebalance/doc/html/api/roadmap.html}{\tt \-A\-P\-I documentation}\/}, which describes all of the modules, classes and functions in the program, is intended as a reference for contributors who are writing code.

\-Force\-Balance is a work in progress; using the program is nontrivial and many features are still being actively developed. \-Thus, users and developers are highly encouraged to contact me through the \href{https://simtk.org/home/forcebalance/}{\tt \-Sim\-T\-K website}, either by sending me email or posting to the public forum, in order to get things up and running.

\-Thanks!

\-Lee-\/\-Ping \-Wang\hypertarget{index_intro_sec}{}\subsection{\-Introduction}\label{index_intro_sec}
\-Welcome to \-Force\-Balance! \-:)

\-This is a {\itshape  theoretical and computational chemistry \/} program primarily developed by \-Lee-\/\-Ping \-Wang. \-The full list of people who made this project possible are given in the \hyperlink{index_credits}{\-Credits}.

\-The function of \-Force\-Balance is {\itshape automatic force field optimization\/}. \-Here \-I will provide some background, which for the sake of brevity and readability will lack precision and details. \-In the future, this documentation will include literature citations which will guide further reading.\hypertarget{index_background}{}\subsubsection{\-Background\-: Empirical Potentials}\label{index_background}
\-In theoretical and computational chemistry, there are many methods for computing the potential energy of a collection of atoms and molecules given their positions in space. \-For a system of {\itshape \-N\/} particles, the potential energy surface (or {\itshape potential\/} for short) is a function of the {\itshape 3\-N\/} variables that specify the atomic coordinates. \-The potential is the foundation for many types of atomistic simulations, including molecular dynamics and \-Monte \-Carlo, which are used to simulate all sorts of chemical and biochemical processes ranging from protein folding and enzyme catalysis to reactions between small molecules in interstellar clouds.

\-The true potential is given by the energy eigenvalue of the time-\/independent \-Schrodinger's equation, but since the exact solution is intractable for virtually all systems of interest, approximate methods are used. \-Some are {\itshape ab initio\/} methods ('from first principles') since they are derived directly from approximating \-Schrodinger's equation; examples include the independent electron approximation (\-Hartree-\/\-Fock) and perturbation theory (\-M\-P2). \-However, most methods contain some tunable constants or {\itshape empirical parameters\/} which are carefully chosen to make the method as accurate as possible. \-Three examples\-: the widely used \-B3\-L\-Y\-P approximation in density functional theory (\-D\-F\-T) contains three parameters, the semiempirical \-P\-M3 method has 10-\/20 parameters per chemical element, and classical force fields have hundreds to thousands of parameters. \-All such formulations require an accurate parameterization to properly describe reality.


\begin{DoxyImage}
\includegraphics[width=10cm]{ladder.png}
\caption{\-An arrangement of simulation methods by accuracy vs. computational cost.}
\end{DoxyImage}


\-The main audience of \-Force\-Balance is the scientific community that uses and develops classical force fields. \-These force fields do not use the \-Schrodinger's equation as a starting point; instead, the potential is entirely specified using elementary mathematical functions. \-Thus, the rigorous physical foundation is sacrificed but the computational cost is reduced by a factor of millions, enabling atomic-\/resolution simulations of large biomolecules on long timescales and allowing the study of problems like protein folding.

\-In classical force fields, relatively few parameters may be determined directly from experiment -\/ for instance, a chemical bond may be described using a harmonic spring with the experimental bond length and vibrational frequency. \-More often there is no experimentally measurable counterpart to a parameter -\/ for example, electrostatic interactions are often described as \-Coulomb interactions between pairs of atomic point \char`\"{}partial charges\char`\"{}, but the fractional charge assigned to each atom has no rigorous experimental of theoretical definition. \-To complicate matters further, most molecular motions arise from a combination of interactions and are sensitive to many parameters at once -\/ for example, the dihedral interaction term is intended to govern torsional motion about a bond, but these motions are modulated by the flexibility of the nearby bond and angle interactions as well as the nonbonded interactions on either side.


\begin{DoxyImage}
\includegraphics[width=10cm]{interactions.png}
\caption{\-An illustration of some interactions typically found in classical force fields.}
\end{DoxyImage}


\-For all of these reasons, force field parameterization is difficult. \-In the current practice, parameters are often determined by fitting to results from other calculations (for example, restrained electrostatic potential fitting (\-R\-E\-S\-P) for determining the partial charges) or chosen so that the simulation results match experimental measurements (for example, adjusting the partial charges on a solvent molecule to reproduce the bulk dielectric constant.) \-Published force fields have been modified by hand over decades to maximize their agreement with experimental observations (for example, adjusting some parameters in order to reproduce particular protein \-N\-M\-R structure) at the expense of reproducibility.\hypertarget{index_mission_statement}{}\subsubsection{\-Purpose and brief description of this program}\label{index_mission_statement}
\-Given this background, \-I can make the following statement. {\bfseries \-The purpose of \-Force\-Balance is to create force fields by applying a highly general and systematic process with explicitly specified input data and optimization methods, paving the way to higher accuracy and improved reproducibility. }

\-At a high level, \-Force\-Balance takes an empirical potential and a set of reference data as inputs, and tunes the parameters such that the simulations are able to reproduce the data as accurately as possible. \-Examples of reference data include energy and forces from high-\/level \-Q\-M calculations, experimentally known molecular properties (e.\-g. polarizabilities and multipole moments), and experimentally measured bulk properties (e.\-g. density and dielectric constant).

\-Force\-Balance presents the problem of potential optimization in a unified and easily extensible framework. \-Since there are many empirical potentials in theoretical chemistry and similarly many types of reference data, significant effort is taken to provide an infrastructure which allows a researcher to fit any type of potential to any type of reference data.

\-Conceptually, a set of reference data (usually a physical quantity of some kind), in combination with a method for computing the corresponding quantity with the force field, is called a {\bfseries target}. \-For example\-:


\begin{DoxyItemize}
\item \-A force field can predict the density of a liquid by running \-N\-P\-T molecular dynamics, and this computed value can be compared against the experimental density.
\end{DoxyItemize}


\begin{DoxyItemize}
\item \-A force field can be used to evaluate the energies and forces at several molecular geometries, and these can be compared against energies and forces from higher-\/level quantum chemistry calculations using these same geometries. \-This is known as {\bfseries force and energy matching}.
\end{DoxyItemize}


\begin{DoxyItemize}
\item \-A force field can predict the multipole moments and polarizabilities of a molecule isolated in vacuum, and these can be compared against experimental measurements.
\end{DoxyItemize}

\-Within a target, the accuracy of the force field can be optimized by tuning the parameters to minimize the difference between the computed and reference quantities. \-One or more targets can be combined to produce an aggregate {\bfseries objective function} whose domain is the {\bfseries parameter space}. \-This objective function, which typically depends on the parameters in a complex way, is minimized using nonlinear optimization algorithms. \-The result is a force field which minimizes the errors for all of the targets.


\begin{DoxyImage}
\includegraphics[height=10cm]{cycle.png}
\caption{\-The division of the potential optimization problem into three parts; the force field, targets and optimization algorithm.}
\end{DoxyImage}


\-The problem is now split into three main components; the force field, the targets, and the optimization algorithm. \-Force\-Balance uses this conceptual division to define three classes with minimal interdependence. \-Thus, if a researcher wishes to explore a new functional form, incorporate a new type of reference data or try a new optimization algorithm, he or she would only need to contribute to one branch of the program without having to restructure the entire code base.

\-The scientific problems and concepts that this program is based upon are further described in my \-Powerpoint presentations and publications, which can be found on the \href{https://simtk.org/home/forcebalance/}{\tt \-Sim\-T\-K website}.\hypertarget{index_credits}{}\subsection{\-Credits}\label{index_credits}

\begin{DoxyItemize}
\item \-Lee-\/\-Ping \-Wang is the principal developer and author.
\end{DoxyItemize}


\begin{DoxyItemize}
\item \-Troy \-Van \-Voorhis provided scientific guidance and many of the central ideas as well as financial support.
\end{DoxyItemize}


\begin{DoxyItemize}
\item \-Jiahao \-Chen contributed the call graph generator, the \-Q\-T\-P\-I\-E fluctuating-\/charge force field (which \-Lee-\/\-Ping implemented into \-G\-R\-O\-M\-A\-C\-S), the interface to the \-M\-O\-P\-A\-C semiempirical code, and many helpful discussions.
\end{DoxyItemize}


\begin{DoxyItemize}
\item \-Arthur \-Vigil contributed the unit testing framework and many unit tests, significant improvements to the automatic documentation generation, logging of output, graphical user interface, and various code improvements.
\end{DoxyItemize}


\begin{DoxyItemize}
\item \-Matt \-Welborn contributed the parallelization-\/over-\/snapshots functionality in the general force matching module.
\end{DoxyItemize}


\begin{DoxyItemize}
\item \-Vijay \-Pande provided scientific guidance and financial support, and through the \-Sim\-Bios program gave this software a home on the \-Web at the \href{https://simtk.org/home/forcebalance/}{\tt \-Sim\-T\-K website}.
\end{DoxyItemize}


\begin{DoxyItemize}
\item \-Todd \-Martinez provided scientific guidance and financial support. 
\end{DoxyItemize}